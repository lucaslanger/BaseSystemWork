\def\year{2015}
\documentclass[letterpaper]{article}

\usepackage{aaai}
\usepackage{times}
\usepackage{helvet}
\usepackage{courier}

\usepackage{graphicx}
\usepackage{amsmath}
\usepackage{amssymb}
\usepackage{amsthm}

\usepackage{algorithm}
\usepackage{algpseudocode}

\frenchspacing
\setlength{\pdfpagewidth}{8.5in}
\setlength{\pdfpageheight}{11in}

\pdfinfo{
/Title (Insert Your Title Here)
/Author (Put All Your Authors Here, Separated by Commas)}

\setcounter{secnumdepth}{0}  

% We define our own macros here
\newcommand{\mat}[1]{\mathbf{#1}}
\newcommand{\A}{\mat{A}}
\newcommand{\B}{\mat{B}}
\newcommand{\M}{\mat{M}}
\renewcommand{\H}{\mat{H}}
\newcommand{\Rset}{\mathbb{R}}
\newcommand{\R}{\Rset}
\newcommand{\Nset}{\mathbb{N}}
\newcommand{\Hset}{\mathbb{H}}
\newcommand{\sstar}{\Sigma^\star}
\newcommand{\Sstar}{\sstar}

\newcommand{\pivec}{\boldsymbol{\pi}}
\newcommand{\alphavec}{\boldsymbol{\alpha}}
\newcommand{\thetavec}{\boldsymbol{\theta}}
\newcommand{\rhovec}{\boldsymbol{\rho}}
\newcommand{\vecalpha}{\alphavec}
\newcommand{\vectheta}{\thetavec}
\newcommand{\vecrho}{\rhovec}
\newcommand{\aone}{\boldsymbol{\alpha}_\lambda}
%\newcommand{\aone}{\boldsymbol{\alpha}_0}
\newcommand{\ainf}{\boldsymbol{\alpha}_{\infty}}
\newcommand{\taone}{\tilde{\boldsymbol{\alpha}}_0}
\newcommand{\tainf}{\tilde{\boldsymbol{\alpha}}_{\infty}}
\newcommand{\haone}{\hat{\boldsymbol{\alpha}}_0}
\newcommand{\hainf}{\hat{\boldsymbol{\alpha}}_{\infty}}
\newcommand{\bone}{\boldsymbol{\beta}_0}
\newcommand{\binf}{\boldsymbol{\beta}_{\infty}}
\newcommand{\vecb}{\boldsymbol{\beta}}
\newcommand{\tbone}{\tilde{\boldsymbol{\beta}}_0}
\newcommand{\tbinf}{\tilde{\boldsymbol{\beta}}_{\infty}}
\newcommand{\hbone}{\hat{\boldsymbol{\beta}}_0}
\newcommand{\hbinf}{\hat{\boldsymbol{\beta}}_{\infty}}
\newcommand{\Alpha}{\aone}
\newcommand{\Beta}{\ainf}
\newcommand{\psrA}{\mathcal{A}}
\newcommand{\psr}{\langle  \aone, \ainf, \{\A_\sigma\} \rangle}
\newcommand{\psrsigma}{\langle \Sigma, \aone, \ainf, \{\A_\sigma\}_{\sigma \in \Sigma} \rangle}
\newcommand{\mpsrsigma}{\langle \Sigma, \Sigma', \kappa, \aone, \ainf, \{\A_\sigma\}_{\sigma \in \Sigma'} \rangle}

%\newcommand{\wb}{\langle  \bone, \binf, \{\B_\sigma\} \rangle}
%\newcommand{\wap}{\langle  \aone', \ainf', \{\A'_\sigma\} \rangle}
%\newcommand{\wbp}{\langle  \bone', \binf', \{\B'_\sigma\} \rangle}
%\newcommand{\wat}{\langle  \taone, \tainf, \{\tilde{\A}_\sigma\} \rangle}
%\newcommand{\wtt}{\langle  \taone, \tainf, \{\tilde{\A}_\sigma^\delta\} \rangle}
%\newcommand{\wt}{\langle  \aone, \ainf, \{{\A}_\sigma^\delta\} \rangle}
%\newcommand{\wbt}{\langle  \tbone, \tbinf, \{\tilde{\B}_\sigma\} \rangle}
%\newcommand{\wah}{\langle  \haone, \hainf, \{\hat{\A}_\sigma\} \rangle}
%\newcommand{\wbh}{\langle  \hbone, \hbinf, \{\hat{\B}_\sigma\} \rangle}

\newcommand{\Bs}{\mathcal{B}}
\newcommand{\cB}{\Bs}
\newcommand{\Ps}{\mathcal{P}}
\newcommand{\Ss}{\mathcal{S}}
\newcommand{\cW}{\mathcal{W}}
\newcommand{\Vsv}{\mat{V}}

\DeclareMathOperator*{\argmax}{\rm argmax}
\DeclareMathOperator*{\argmin}{\rm argmin}
\DeclareMathOperator{\sgn}{sgn}
\DeclareMathOperator{\sign}{sign}
\DeclareMathOperator{\range}{range}
\DeclareMathOperator{\rank}{rank}
\DeclareMathOperator{\diag}{diag}
\DeclareMathOperator{\local}{local}
\DeclareMathOperator{\Tr}{Tr}

\usepackage[textsize=footnotesize,color=green!40]{todonotes}
\newcommand{\borja}[1]{\todo[inline,color=green!40]{{\it Borja:~}#1}}
\newcommand{\doina}[1]{\todo[inline,color=red!40]{{\it Doina:~}#1}}
\newcommand{\lucas}[1]{\todo[inline,color=blue!40]{{\it Lucas:~}#1}}

\begin{document}

\title{TITLE GOES HERE}
\author{Anonymous Submission}
\maketitle

\begin{abstract}
\begin{quote}
A standard problem in machine learning is making accurate predictions in partially observable environments. Spectral algorithms which learn PSRs offer statistical consistency, but often have trouble with computational efficiency as the models they learn are too large. In practice, one solves this computational issue by truncating the original model, eliminating the weakest states. Despite this, performance with reduced models is often quite poor. With this issue in mind, we develop a novel extension to PSRs which we call Multi-PSRs. In addition, we provide two algorithms for M-PSRs, one for learning parameters and the other for making effective queries. The M-PSR leverages structure in observations sequences and expresses queries more compactly. This extended model shares all the benefits of the traditional PSR but performs far better for smaller models in experiments. We perform experiments of robot exploration in labyrinth environments and cover both the single observation case (timing) and the multiple observation case. We pick these environments as PSRs have been used for planning applications on these kinds of environments in the past (CITE). In our experiments, we show that the improvement of M-PSRs hold for varying amounts of data, environment sizes, and number of observations symbols.
\end{quote}
\end{abstract}

\section{Introduction}

Learning models of partially observable dynamical systems is a problem that arises in many practical applications, including robotics, medical monitoring, market analysis etc.  Predictive state representations (PSRs)~\cite{littmanpsr,singh04,rosencrantz04}, provide a theoretically justified approach in which at any time-step the current state is modeled as a set of predictions about the future evolution of the system, conditioned on past outcomes.  The most popular of these models are linear PSRs, in which predictions for any future trajectories can be obtained as a linear combination of a finite set of ``core" predictions. A variety of spectral algorithms for learning linear PSRs have been proposed in recent works, e.g.~\cite{bootspsr,Hamilton2013}.

Spectral algorithms are appealing because of their strong theoretical properties, such as statistical consistency and learning efficiency~\cite{hsu09,bailly10}. Unfortunately, their practical uptake is still limited, in part, due to the fact that these general-purpose algorithms are not designed to leverage the structural regularities frequently found in applications. This is a challenging question that needs to be tackled in order to facilitate more efficient spectral learning algorithms for specific applications. Recent work along these lines has focused on special structures encountered in epigenomics \cite{zhang2015spectral}. In this paper, we focus on a specific type of structure which arises frequently in dynamical systems: behaviour which takes place at different time scales. This type of structure is often leveraged to provide efficient algorithms in signal processing. Similar gains can be obtained by using multiple time scales in planning and reinforcement learning, e.g.~\cite{sutton99}.

Our approach is based on a new class of PSRs which we call the multi-step PSR (M-PSR). Like the classical linear PSR, M-PSRs represent the dynamics of a partially observable system with discrete observations. However, unlike PSRs, M-PSRs are able to capture structure occurring at multiple time scales by representing transitions between states over multi-step observations. Our main result is a spectral learning algorithm for M-PSRs combined with a data-driven strategy for multi-step observation selection. Through an extensive empirical evaluation, we show that in environments where characteristic multi-step observations occur frequently, M-PSRs improve the quality of learning with respect to classical PSRs. This improvement is uniform over a range of environment sizes, number of observation symbols, and amounts of training data. 

\section{M-PSR: Definition and Learning}

\subsection{Multi-Step PSR}

A linear \emph{predictive state representation} for an autonomous dynamical system with discrete observations in a set $\Sigma$ is a tuple $\psrA = \psrsigma$ where: BLA BLA.

To define our model for multi-step PSR we basically augment a PSR with two extra objects: a set of \emph{multi-step observations} $\Sigma' \subset \Sigma^+$ containing non-empty strings formed by basic observations, and a \emph{coding function} $\kappa : \Sigma^\star \to {\Sigma'}^{\star}$ that given a string of basic observations produces an equivalent string composed using multi-step observations.
%
The choice of $\Sigma'$ and $\kappa$ can be quite application-dependent, in order to reflect the particular patterns arising from different environments. However, we assume this objects satisfy a basic set of requirements for the sake of simplicity and to avoid degenerate situations:
\begin{enumerate}
\item The set $\Sigma'$ must contain all symbols in $\Sigma$; i.e.\ $\Sigma \subseteq \Sigma'$
\item The function $\kappa$ satisfies $\partial(\kappa(x)) = x$ for all $x \in \Sigma^\star$, where $\partial : {\Sigma'}^\star \to \Sigma^\star$ is the \emph{decoding morphism} between free monoids given by $\partial(z) = z \in \Sigma^\star$ for all $z \in \Sigma'$. Note this implies that $\kappa(\epsilon) = \epsilon$, $\kappa(\sigma) = \sigma$ for all $\sigma \in \Sigma$, and $\kappa$ is injective.
\end{enumerate}

Using these definitions, a \emph{multi-step PSR} (M-PSR) is a tuple $\psrA' = \mpsrsigma$.

\subsection{Spectral Learning Algorithm}

We extend \cite{bootspsr}...

\subsection{Examples}

%\textbf{Base PSR for Duration Models} BLA BLA...

In this section we will go through a few examples of Multi-PSRs which are particularly relevant to our experiment section. 

For the timing case one can use a M-PSR which we call the \textbf{Base System}. For this type of M-PSR, $\sum'$ will consist of $\{a^{2^k} \forall k <= n\}$. To describe the encoding map we write an observation $a^m = a^{2^n1} ... \cdot a^{2^nf}$  With this equality a natural encoding map $\kappa$ which is natural to use is: $\kappa(a^n) = \{a^{2^n_1},...,\{a^{2^n_f}\}$. 

For the multiple observation case one can also use The Base System. In this case $\sum' = \{\sigma^{2^k}\forall \sigma \in \sum, \forall k <= n\}$. For the encoding map $\kappa$ we first split the string into sequences of a fixed symbol and then use the same encoding as for timing. The Base System for multiple observations clearly aims at environments where observation symbols come in streaks.
 
Another example for the multiple observation case is an M-PSR we will call the \textbf{Tree System}. For the Tree System, we set $\sum'=$ the set of all possible strings of length $<= L$. For the decoding map $\kappa$, we first split a string x into $x_1x_2...x_ny$, where $xi$ $\forall i<=n$ have length L and y has length $len(x) - n \times L$. With this we set $\kappa(x) = \{x_1,x_2,...,y\}$.

The the construction of the above multi-PSRs is not dependent on the environment. In practice, if one would like to learn a M-PSR which has the best performance, namely a M-PSR whose parameters best reflect the types of observations seen from one's environment. 

\section{Learning Algorithm for M-PSR}

In this section, we describe a learning algorithm for M-PSR which combines the standard spectral algorithm for PSR \cite{bootspsr} with a data-driven greedy algorithm for building an extended set of symbols $\Sigma'$ containing frequent patterns that minimise a coding cost for a general choice of coding function $\kappa$.

\subsection{Spectral Learning Algorithm}

We extend \cite{bootspsr} to M-PSR under the assumption that $\kappa$ and $\Sigma'$ are given.

\borja{Need to fill this. Probably copy-paste from the ODM paper will do.}

\borja{BEGIN OF COPY-PASTE. NEEDS EDITS}

A convenient algebraic way to summarize all the information conveyed by $f$ is
with its \emph{Hankel matrix}, a bi-infinite matrix $\H_f \in \R^{\Sstar \times \Sstar}$ with rows and
columns indexed by strings in $\Sstar$.
%
Strings indexing rows and columns are interpreted as prefixes and suffixes respectively.
The entries in $\H_f$ are given by $\H_f(u,v) = f(u,v)$ for every $u, v
\in \Sstar$.

% THIS IS FOR THE SDM
%$\H_f(u,v) = f_u(v) = f(u v) / f(u)$,
%\textbf{TODO (Borja):} Beware the case $f(u) = 0$ !!

Although $\H_f$ is an infinite matrix, in some cases it can have finite rank.
%
In particular, a well-known result states that $\H_f$ has rank at most $n$ if
and only if  there exists a PSR $\psrA$ with $n$ states satisfying
$f_{\psrA} = f$ \cite{CarlylePaz71,Fliess74}.
%
This result is the basis of recently developed spectral learning algorithms for
PSRs \cite{bootspsr}, which we review in Sec.~\ref{sec:learning}.

Instead of looking at the full Hankel matrix, algorithms usually work with finite
sub-blocks of this matrix.
%
A convenient way to specify such blocks is to give the ``names'' to the rows and
columns.
%
Specifically, given a finite set of prefixes $\Ps \subset \Sstar$ and a finite
set of suffixes $\Ss \subset \Sstar$, the pair $\Bs = (\Ps,\Ss)$ is
a \emph{basis} defining the sub-block $\H_\Bs \in \R^{\Ps \times \Ss}$ of
$\H_f$, whose entries are given by $\H_\Bs(u,v) = \H_f(u,v)$.
% for every $u \in
%\Ps$ and $v \in \Ss$.
%
Note that every sub-block built in this way satisfies $\rank(\H_\Bs) \leq
\rank(\H_f)$; when equality is attained, the basis $\Bs$ is
\emph{complete}.

Sometimes it is also convenient to look at one-step shifts of the finite
Hankel matrices.
%
Let $\H \in \R^{\Ps \times \Ss}$ be a finite sub-block of
$\H_f$ specified by a basis $\Bs = (\Ps,\Ss)$.
%
Then, for every symbol $\sigma \in \Sigma$, we define the sub-block $\H_\sigma \in
\R^{\Ps \times \Ss}$ whose entries are given by $\H_\sigma(u,v) = \H_f(u,\sigma
v)$.
%
For a fixed basis, we also consider the vectors $\mat{h}_{\Ss}
\in \R^{\Ss}$ with entries given by $\mat{h}_{\Ss}(v) = \H_f(\lambda,v)$
for every $v \in \Ss$, and $\mat{h}_{\Ps} \in \R^{\Ps}$ with $\mat{h}_{\Ps}(u) =
\H_f(u,\lambda)$.

The Hankel matrix $\H_f$ is tightly related to the \emph{system dynamics matrix}
(SDM) of the stochastic process described by $f$~\cite{singh04}, but while the entries of the
Hankel matrix represent \emph{joint} probabilities over prefixes and suffixes,
the corresponding entry in the SDM is the \emph{conditional} probability of
observing a suffix given the prefix.


The algorithm takes as input $\Sigma$ and a basis
$\Bs$ in $\Sstar$, uses them to estimate the corresponding Hankel matrices, and
then recovers a PSR by performing singular value decomposition and
linear algebra operations on these matrices.
%
%Estimating Hankel matrices containing probabilities of observed trajectories
%from a sample is straightforward, and the details of the spectral learning
%algorithm are reviewed below.
%
Although the method works almost out-of-the-box, in practice the results tend
to be sensitive to the choice of basis.
%
Thus, after briefly recapitulating how the spectral learning algorithm proceeds,
we will devote the rest of the section to describe a procedure for building a
basis which is tailored for the case of learning option duration models.

%\textbf{TODO: Now we should talk in here about Hankel matrices}

Suppose the basis $\Bs$ is fixed and the desired number of states $n$ is given.
%
Suppose that a set of sampled trajectories was used to estimate the
Hankel matrices $\H, \H_\sigma \in \R^{\Ps \times \Ss}$ and vectors
$\mat{h}_{\Ps} \in \R^{\Ps}$, $\mat{h}_{\Ss} \in \R^{\Ss}$ defined in
Sec.~\ref{sec:hankel}.
%
The algorithm starts by taking the truncated SVD $\mat{U}_n \mat{D}_n
\mat{V}_n^\top$ of $\H$, where $\mat{D}_n \in \R^{n \times n}$ contains the first
$n$ singular values of $\H$, and $\mat{U}_n \in \R^{\Ps \times n}$ and $\mat{V}_n
\in \R^{\Ss \times n}$ contain the first left and right singular vectors
respectively.
%
Finally, we compute the transition operators of a PSR as $\A_\sigma =
\mat{D}_n^{-1} \mat{U}_n^\top \H_\sigma \mat{V}_n$, and the initial and final
weights as $\aone^\top = \mat{h}_{\Ss}^\top \mat{V}_n$ and $\ainf =
\mat{D}_n^{-1} \mat{U}_n^\top \mat{h}_{\Ps}$.
%
This yields a PSR with $n$ states. It was proved in~\cite{bootspsr} this
algorithm is statistically consistent: if the population Hankel matrices are
known and the basis $\Bs$ is complete, then the learned PSR is equivalent to the
one that generated the data.


\borja{END OF COPY-PASTE}

\subsection{Notation}

\borja{We should move this into the two subsections below}

\textbf{Obs}: A mapping from observation sequences to the number of occurrences of that sequence in one's dataset. 

\textbf{SubObs} : all substrings of \textbf{Obs}.

\textbf{Q}: A query string (a string for which one wishes to determine the probability of).

\textbf{bestE}: A map from indices i of Q to the optimal encoding of Q[:i].

\textbf{minE}: A map from indices i of Q to $|bestE[i]|$

\textbf{opEnd}: A map from indices i of Q to the set of strings in $\Sigma'$: $\{x \in \Sigma' s.t Q[i-x.length:i] == x\}$

\textbf{numOps}: The desired number of operators one wants in $\Sigma'$. I.e numOps =  $|\Sigma'|$

\subsection{A General Coding Function}

Here we provide a dynamic programming algorithm which can serve as $\kappa$ for any M-PSR. Given a query string Q, and a set of transition sequences $\Sigma'$, the algorithm minimizes the number of sequences used in the partition $\kappa(Q)$. In other words, the algorithm minimizes $|\kappa(Q)|$. For the single observation case, the algorithm is equivalent to the coin change problem.

For a given string Q, the algorithm inductively computes the optimal string encoding for the prefix Q[:i]. It does so by minimizing over all $s \in \Sigma'$ which terminate at the index i of Q.

\begin{algorithm}
\caption{Encoding Algorithm}
\label{Encoding Algorithm}
\begin{algorithmic}[1]
\Procedure{DPEncode}{}

\State $bestE[] \gets new String[len(Q)+1]$
\State $minE[] \gets new Int[len(Q)+1]$
\State $opEnd[] \gets new String[len(Q)+1][]$

\State $bestEnd[0] = Q[0]$
\State $minE[0] = 0$

\For{i in range[1,Q.length]}
	 \State $opEnd[i] \gets \{s \in \Sigma', Q[i-len(s):i] == s\}$
\EndFor

\For{i in range[1,Q.length]}
	\State $bestOp \gets null$
	\State $m \gets null$ 
	\For{$s \in opEnd[i]$}
		\State $tempInt \gets minE[i-len(s)] + 1$
		\If{$m == null$ or $tempInt < m$}
			\State $m \gets temp$ 
			\State $bestOp \gets s$
		\EndIf
	\EndFor
	\State $minE[i+1] \gets m$
	\State $bestE[i+1] \gets bestE[i-len(bestOp)] + bestOp$
\EndFor

\Return $bestE[len(Q)]$

\EndProcedure
\end{algorithmic}
\end{algorithm}

\subsection{Greedy Selection of Multi-Step Observations}

Here we present a greedy heuristic which learns the multi-step transition sequences $\Sigma'$ from observation data. Having a $\Sigma'$ which reflects the types of observations produces by one's system will allow of short encodings when coupled with our a dynamic programming encoding algorithm. In practice, this greedy algorithm will pick substrings from one's observation set which are long, frequent, and diverse. From an intuitive standpoint, one can view structure in observation sequences as relating to the level of entropy in the system's observations. 

As a preprocessing step, we reduce the space of substrings textbf{subObs} to the k most frequent substrings in our observation set. Here frequent means the number of observation sequences a given substring $s \in subObs$ occurs in. 
\lucas{Added substring preprocessing step}

The algorithm evaluates substrings based on how much they reduce the number of transition operators used on one's observation data. The algorithm adds the best operator iteratively with $\Sigma'$ initialized to $\Sigma$. More formally at the i'th iteration of the algorithm the following is computed: $min_{sub \in SubObs} \sum\nolimits_{obs \in Obs}|\kappa(obs,\Sigma'_i \cup sub)|$. The algorithm terminates after the \textbf{numOps} iterations. 

\begin{algorithm}
\caption{Base Selection Algorithm}
\label{Base Selection}
\begin{algorithmic}[1]
\Procedure{Base Selection}{}
\State $\Sigma' \gets \{s, s \in \sum \}$
\State $Subs \gets \{$k frequent $s \in subObs\}$

\State $prevBestE \gets null$
\For{each obs in Obs}
	\State $prevBestEncoding[obs] \gets len(obs)$
\EndFor

\State $i\gets 0$\
\While{$i<numOperators$}
	\State $bestOp \gets null$
	\State $bestImp \gets null$
	\For{each s $\in Subs$ }
		\State $c \gets 0$
		\For{each obs in Obs}
			\State $c \gets c+DPEncode(obs)-prevBestE(obs)$
		\EndFor
		
		\If{$c>bestImp$}
			\State $bestOp \gets observation$
			\State $bestImp \gets c$
		\EndIf
		
	\EndFor

	\State $\Sigma' \gets \Sigma' \cup bestOp$
	\For{each obs in Obs}
		\State $prevBestE \gets DPEncode(obs,\Sigma'$) 
	\EndFor	
	
	\State $i \gets i + 1$
\EndWhile 
\Return $\Sigma'$

\EndProcedure
\end{algorithmic}
\end{algorithm}


\section{Experiments}

We assess the performance of the Base System on labyrinth environments. The robot is positioned in a starting location and it stochastically navigates the environment based on transition probabilities from states to states. We split the experiments into two cases. The first is the case of timing, where $\sum = \{\sigma\}$ and the second is with multiple observations. For timing, the goal is to make predictions about how long the agent will survive the environment. One can also ask conditional queries such as how long the agent should expect to survive given that t seconds have elapsed $f(\sigma^m|\sigma^n) = (eqns)$. For multiple observations we place the agent in the environment, let it transition between states for a fixed number of observations and then remove the agent. The goal for the multiple observation labyrinth will be to make predictions about seeing observation sequences. In both cases, we analyse the performance for M-PSRs of different model sizes with a fixed observation set. For each Base System M-PSR, we include all powers of 2 up to 256. For the heuristic based PSRs we learn 10 operators from the observations with the greedy learning algorithm for obtaining $\kappa$ described in (). To measure the performance of a PSR we use the following norm:
$||f - \hat{f}|| = \sqrt{\sum\nolimits_{x \in observations}(f(x) - \hat{f(x)})^2}$. We use this norm because of a bound presented by [AUTHORS], which states that (). The function f is obtainable explicitly for the environments which we consider as we have access to the underlying HMM of the environment. We compute approximations to this error norm, by fixing a set of strings T and summing over T. For the timing case, we take T to be the $\{sigma^k, k<=n\}$, while for the multiple observation case, we take all possible strings producible from the prefixes and suffixes in our dataset. That is, for the multiple observation case $T = \{p \cdot c, \forall p \in P, s \in S\}$.

\subsection{Learning PSRs for Timing}
For the timing case, we construct our empirical hankel matrix by including ${\sigma^i i<=n}$. With this choice, the empirical hankel matrix will be a nxn matrix with the prefixes and suffixes being the same.The parameter n depends on the application. For Double Loop environments we set n to be 300, while for pacMan it was 600. The important property that needs to be satisfied when choosing the parameter n is that enough observations are captured to learn a good model. Something here: $\lim_{ \to 2} f(x) = 5$ .

\subsection{Learning PSRs for Multiple Observations}

For multiple observations a slightly more complex approach is required to construct the empirical hankel matrix. For prefixes, we select the k most frequent prefixes from our observations set. For suffixes we take all suffixes that occur from our set of prefixes. This heuristic was given in previous work by [] and it showed that ().

\subsection{Double Loop Timing}

For timing, we start by considering a double loop environment. The lengths of the loops correspond to the number of states in the loop. Every time the agent transitions to a new state, a $sigma$ observation is produced.  An observation of $\sigma^{100}$ corresponds to the agent surviving the environment for 100 time units. The agent starts at the intersection of the two loops. At the intersection, the agent has a 50 percent chance of entering either loop. At intermediate states in the loops it simply moves to the next state in the loop with probability p and remains in its current state with probability 1-p. Exit states are located halfway between each loop. At an exit state, the agent has a 50 percent probability of exiting the environment. In figure 1, we learn a PSRs with 10000 observation sequences. We generate 10 PSRs and average their performance. 

\begin{figure}[ht!]
\centering
\includegraphics[width=60mm]{uCOREPICS/DoubleLoopTimingHeuristics47-27.png}
\caption{Double Loop Environment\label{overflow}}
\end{figure}

\begin{figure}[ht!]
\centering
\includegraphics[width=60mm]{uCOREPICS/DoubleLoop64-16Heuristics.png}
\caption{Double Loop Environment\label{overflow}}
\end{figure}

\subsection{Pacman Timing}

We proceed to work with timing in a Pacman environment. The transition structure of this environment is shown in Figure 2. Edge weights vary from 1 to 3 and are stretched by a parameter which we call the stretch factor. Again we use 10000 observation sequences per PSR and compute the average error of 10 PSRs.

\begin{figure}[ht!]
\centering
\includegraphics[width=60mm]{uCOREPICS/PacManTimingHeuristicsIncluded.png}
\caption{Double Loop Environment\label{overflow}}
\end{figure}

\subsection{Timing - Results}
As is demonstrated in figure 3, learning longer transitions provides a significant improvement over the standard for truncated models. The heuristic based approach performs slightly better than the powers of two method. For the double loop case the heuristic approach learns multiples of the loop lengths which results in partitions which use fewer operators. For Pacman, the operators that are learned are various multiples of the stretch factor, which once again shows that the greedy heuristic is effective. 

\subsection{Multiple Observations}

To test whether the Base System would translate to multiple observations we construct a double loop environment where one loop is green and the other is blue. The lengths of each loop are also varied. We fix the length of observations to be $loop1 + loop2 * 3$. To build empirical estimates of probabilities we set f(x)=prefix-occ(x)/num-strings-length>=x.

\subsection{Multiple Observations Results}

\begin{figure}[ht!]
\centering
\includegraphics[width=60mm]{uCOREPICS/PacManTimingHeuristicsIncluded.png}
\caption{Double Loop Environment\label{overflow}}
\end{figure}


\section{Conclusion}

\section{Acknowledgments}
Funding and friends...

\bibliographystyle{aaai}
\bibliography{references}

\end{document}
