\section{Introduction}

Learning models of partially observable dynamical systems is a problem that arises in many practical applications, including robotics, medical monitoring, market analysis etc.  Predictive state representations (PSRs)~\cite{littmanpsr,singh04,rosencrantz04}, provide a theoretically justified approach in which at any time-step the current state is modeled as a set of predictions about the future evolution of the system, conditioned on past outcomes.  The most popular of these models are linear PSRs, in which predictions for any future trajectories can be obtained as a linear combination of a finite set of ``core" predictions. A variety of spectral algorithms for learning linear PSRs have been proposed in recent works, e.g.~\cite{bootspsr,Hamilton2013}.

Spectral algorithms are appealing because of their strong theoretical properties, such as statistical consistency and learning efficiency~\cite{hsu09,bailly10}. Unfortunately, their practical uptake is still limited, in part, due to the fact that these general-purpose algorithms are not designed to leverage the structural regularities frequently found in applications. This is a challenging question that needs to be tackled in order to facilitate more efficient spectral learning algorithms for specific applications. Recent work along these lines has focused on special structures encountered in epigenomics \cite{zhang2015spectral}. In this paper, we focus on a specific type of structure which arises frequently in dynamical systems: behaviour which takes place at different time scales. This type of structure is often leveraged to provide efficient algorithms in signal processing. Similar gains can be obtained by using multiple time scales in planning and reinforcement learning, e.g.~\cite{sutton99}.

Our approach is based on a new class of PSRs which we call the multi-step PSR (M-PSR). Like the classical linear PSR, M-PSRs represent the dynamics of a partially observable system with discrete observations. However, unlike PSRs, M-PSRs are able to capture structure occurring at multiple time scales by representing transitions between states over multi-step observations. Our main result is a spectral learning algorithm for M-PSRs combined with a data-driven strategy for multi-step observation selection. Through an extensive empirical evaluation, we show that in environments where characteristic multi-step observations occur frequently, M-PSRs improve the quality of learning with respect to classical PSRs. This improvement is uniform over a range of environment sizes, number of observation symbols, and amounts of training data. 
