\section{Discussion}

We presented an approach to leveraging multiple temporal scales in the behaviour of dynamical systems, in order to learn predictive state representations.  The proposed model, M-PSR, can be learned by adapting existing spectral approaches.  To our knowledge, the only other work that attempts to include temporal abstraction in PSRs is due to ~\cite{wolfe06}, who's use temporally extended actions, or options~\cite{sutton99}, and build PSRs on top of these.  However, this model has a very different flavour from our approach, as we bundle together observations into multiple steps.  In particular, our approach is applicable even when there are no actions, such as in the case of HMMs, whereas this previous work requires structure in the action space.
In all the experiments we conducted, M-PSRs offer significantly better predictive performance for reduced model sizes than PSRs. In addition, Data-Driven PSRs offer improvements over generic M-PSRs by learning the transition operators specific to the environment, which is very important when prior information about appropriate time scales for modelling is not known.  Our evaluation focused on illustrating the advantage of the proposed model especially when the amount of data available is small, and in noisy environments.
 Although this was not specifically illustrated in the experiments. M-PSRs offer a computational advantage when performing conditional probability queries as they use far fewer matrices than regular PSRs. This can be very important for online applications, such as in planning environments. 
 
 We presented a comprehensive set of experiments in simulated domains, in order to have some ground truth available.  However, we anticipate that the proposed models would be useful in real tasks, for example in financial market prediction.  In this case, one could discretize market levels into bins, and use M-PSRs to learn predictive models over multiple time steps, in order to capture both long-term trends and fast fluctuations.  We also plan to experiment with this approach in the domain of processing physiological signal recordings, which similarly exhibit short and long-term trends.  On the theoretical side, it would be interesting to analyze the existence of an ``optimal" set of symbols for the data-driven M-PSR, and to develop further algorithms for finding it.

